\documentclass[a4paper,12pt]{letter}
\usepackage[T1]{fontenc} 
\usepackage[utf8]{inputenc}
\usepackage[italian]{babel} 
\usepackage{eurosym}

%% Template: determinaB.tex - vers. 1.2

((* include 'header.tex' *))

\begin{document}
\topaddr

{\small Pratica N. ((( pratica.numero_pratica ))) }
\vspace{1cm}

\begin{flushright}

Determina n. ((( pratica.numero_determina_b )))
\end{flushright}

{\bf Oggetto:} Procedura negoziata per  Fornitura/Servizio/Lavoro di: ((( pratica.descrizione_acquisto ))),
indetta con determinazione  n. ((( pratica.numero_determina ))). Aggiudicazione definitiva.

\begin{center}
Il Direttore
\end{center}

\vspace{0.5 cm}

\begin{list}{RISCONTRATA}{}


\item[VISTO] il Decreto Legislativo 4 giugno 2003, n. 138 di riordino dell'INAF;
\item[VISTO] il nuovo Statuto dell'INAF che è stato definitivamente approvato dal
             C.d.A. con delibera numero 42 del 25.05.2018, pubblicato sul sito web
             istituzionale in data 7 settembre 2018 ed entrato in vigore il 24 settembre 2018;
\item[VISTO]  il Decreto Legislativo n. 50/2016 del 18/04/2016 e s.m.i., ed in particolare
             gli articoli 36 e 95;
\item[VISTO] il D.P.R. n. 207 del 5 ottobre 2010 recante “Regolamento di esecuzione
             ed attuazione del D.Lgs. 163/2006” nelle parti ancora in vigore;
\item[VISTE] le linee guida ANAC;
\item[VISTA]  la L. 241/1990 che stabilisce che: "L'attività amministrativa persegue
             i fini determinati dalla legge ed è retta da criteri di economicità, di
             efficacia, di pubblicità e di trasparenza";
\item[VISTO]  il Decreto Legislativo 25 novembre 2016 n. 218 ed in particolare l'Art. 10;
\item[VISTO]  il D.L. n. 52 del 7 maggio 2012, trasformato in Legge n. 94 del 6 luglio
             2012 recante Disposizioni urgenti per la razionalizzazione della spesa
             pubblica e la successiva Legge135 del 7 agosto 2012 Disposizioni urgenti per
             la revisione della spesa pubblica;
\item[VISTO] il Disciplinare di Organizzazione e Funzionamento INAF approvato con delibera
             del C.D.A n.44/2012 del 21/06/2012;
\item[VISTO] il Regolamento sull'Amministrazione, sulla Contabilità e sull'Attività
             Contrattuale dell'INAF, pubblicato sul S.O. n. 185 alla G.U. Serie Generale
             n.~300 del 23 dicembre 2004;
\item[VISTA] la determina N. ((( pratica.numero_determina ))) del ((( pratica.data_determina )))
             con cui è stata avviata una procedura negoziata per Fornitura/Servizio/Lavoro di:
             ((( pratica.descrizione_acquisto ))) e con la quale è stato nominato il R.U.P.;
\item[CONSIDERATO]  che per la suddetta procedura è stato adottato quale criterio di
              aggiudicazione, quello %
((* if pratica.criterio_assegnazione == 'prezzo.piu.basso' *)) %
  del prezzo più basso;
((* else *)) %
  dell'offerta più vantaggiosa;
((* endif *))

((* if pratica.str_oneri_sicurezza *))
\item[RILEVATO] che nel caso di specie gli oneri per la sicurezza sono valutati congrui
                nell'importo di € ((( info.str_oneri_sicurezza )));
((* endif *))
\item[CONSIDERATO] che il Responsabile Unico del Procedimento, ha invitato alla predetta
                   procedura di gara gli operatori economici elencati nella lettera di invito allegata;

((* if pratica.ditta_vincitrice *))
\item[CONSIDERATO] che, entro il termine stabilito dalla predetta lettera di invito,
                   ovvero le ore ((( pratica.fine_gara_ore ))) del giorno
                   ((( pratica.fine_gara_giorno ))) sono pervenute le offerte degli
                   operatori economici di seguito elencati:
\begin{itemize}
((* for i in pratica.lista_ditte *))
    \item ((( i.nome_ditta ))) - ((( i.sede_ditta )))
((* endfor *))
\end{itemize}

\item[RILEVATO] che l'offerta della ditta ((( pratica.ditta_vincitrice.nome_ditta ))) - ((( pratica.ditta_vincitrice.sede_ditta ))) risulta
                prima nella graduatoria della gara;
\item[CONSTATATO] che si sono conclusi con esito positivo i controlli di rito sul 
                  possesso dei requisiti di ordine generale dichiarati;
\item[RISCONTRATA]  la regolarità della procedura di gara fin qui seguita;
\item[CONFERMATA]  la copertura finanziaria sul/sui fondo/i ((( pratica.stringa_codf )));
\item[RITENUTO]  che vi siano i presupposti normativi e di fatto per l'affidamento in oggetto;
\end{list}

\begin{center}
Determina
\end{center}

\begin{list}{VALUTATA}{}

\item[Art. 1:]	di procedere all'aggiudicazione definitiva a favore della ditta
                ((( pratica.ditta_vincitrice.nome_ditta ))) - ((( pratica.ditta_vincitrice.sede_ditta )))
		della procedura negoziata per quanto
                riportato nell'oggetto della determina iniziale;
\item[Art. 2:] 	di stabilire nell'importo di ((( pratica.str_prezzo_gara )))((* if pratica.str_oneri_it *)),
		di cui ((( pratica.str_oneri_it ))) per oneri di sicurezza, ((* endif *))
                il corrispettivo contrattuale alla
                luce dell'offerta presentata in sede di gara da parte della ditta su indicata;
\item[Art. 3:]	di prenotare l'importo complessivo a favore della ditta
                ((( pratica.ditta_vincitrice.nome_ditta )))  con sede legale in
		((( pratica.ditta_vincitrice.sede_ditta )))
                imputando la spesa sui fondi ((( pratica.stringa_codf )));
\item[Art. 4:] 	L'importo del contratto dovrà essere impegnato sulla competenza dell'esercizio
                su cui graverà il costo dello stesso.
\end{list}

((* else *))
\item[RILEVATO] che non è pervenuta alcuna offerta rispondente ai termini di gara;
\end{list}

\begin{center}
Determina
\end{center}

\begin{list}{VALUTATA}{}
\item[Art. 1:]	di dichiarare nulla la procedura di gara

((* if pratica.art_2 *))
\item[Art. 2:] ((( pratica.art_2 )))
((* endif *))

\end{list}
((* endif *))
\vspace{0.5cm}

\begin{minipage}{\linewidth}
Firenze, lì ((( pratica.data_determina_b )))

\begin{flushright}
\begin{minipage}[t]{6cm}
\begin{center}
Il Direttore \\
((( pratica.titolo_direttore ))) ((( pratica.nome_direttore )))
\end{center}
\end{minipage}
\end{flushright}
\end{minipage}
\end{document}
