\documentclass[a4paper,12pt]{letter}
\usepackage[T1]{fontenc} 
\usepackage[utf8]{inputenc}
\usepackage[italian]{babel} 
\usepackage{eurosym}

%% Template: determina_inf5000.tex - vers. 1.0 

((* include 'header.tex' *))

\begin{document}
\topaddr

{\small Pratica N. ((( pratica.numero_pratica ))) }
\vspace{1cm}

\begin{flushright}

Determina n. ((( pratica.numero_determina )))
\end{flushright}

Determina a contrattare del ((( pratica.data_determina ))) - Fornitura/Servizio/Lavoro di: ((( pratica.descrizione_acquisto ))).

Motivazione: ((( pratica.motivazione_acquisto )))
\vspace{0.5cm}

\begin{center}
Il Direttore
\end{center}

\vspace{0.5 cm}

\begin{list}{VALUTATA}{}
\item[VISTO] il Decreto Legislativo 4 giugno 2003, n. 138 di riordino dell'INAF;
\item[VISTO] il Decreto Legislativo 25 novembre 2016 n. 218 ed in particolare l'Art. 10;
\item[VISTA] la L. 244 del 24 dicembre 2007 recante ``Disposizioni per la formazione 
        del bilancio annuale e pluriennale dello Stato (legge finanziaria 2008)''; 
\item[VISTO]  il Decreto Legislativo n. 50/2016 del 18 aprile 2016 ed in particolare
      l'Art. 36;
\item[VISTO] il D.P.R. n. 207 del 5 ottobre 2010 recante ``Regolamento di esecuzione 
        ed attuazione del D.Lgs. 163/2006'' nelle parti ancora in vigore; 
\item[PRESO ATTO] delle linee guida dell'ANAC;
\item[VISTA] la L. 241/1990 che stabilisce che: ``L'attività amministrativa persegue i 
        fini determinati dalla legge ed à retta da criteri di economicità, di 
        efficacia, di pubblicità e di trasparenza''
\item[VISTO] il D.L. n. 52 del 7 maggio 2012, trasformato in Legge n. 94 del 6 luglio 2012 
        recante "Disposizioni urgenti per la razionalizzazione della spesa pubblica 
        e la successiva Legge 135 del 7 agosto 2012 "Disposizioni urgenti per la 
        revisione della spesa pubblica"; 
\item[VISTO] il Disciplinare di Organizzazione e Funzionamento dell'INAF, approvato
        con delibera del C.D.A. n. 44/2012 del 21/06/2012;
\item[VISTO] il Regolamento sull'Amministrazione, sulla Contabilità e sull'Attività 
        Contrattuale dell'INAF, pubblicato sul S.O. n. 185 alla G.U. Serie Generale 
        n. 300 del 23 dicembre 2004;   
\item[VISTA] la richiesta da parte di ((( pratica.nome_responsabile ))) di  acquisire:
         ((( pratica.descrizione_acquisto ))) \\
\item[VISTA] la Legge 136 Art. 3 del 13/8/10 e il D.L. n. 187/2010 convertito nella Legge 
        n. 217 del 17.12.2010, che introducono l'obbligo di tracciabilità dei flussi 
        finanziari relativi alle commesse pubbliche; 
\item[VISTA] la disponibilità sul/sui fondo/i ((( pratica.stringa_codf ))); 
\item[RITENUTO] quindi che vi siano i presupposti normativi e di fatto per acquisire  
         quanto indicato in oggetto ai sensi dell'Art. 36 co. 2 lett. A del D.Lgs. 50/2016 e s.m.i.;
\item[VALUTATA] la necessità di provvedere all'acquisizione di quanto richiesto; 
\end{list}



\begin{center}
DETERMINA
\end{center}

\begin{list}{VALUTATA}{}
\item[Art.~1:] di nominare, quale responsabile unico del procedimento, in base 
           all'Art. 31 del D.Lgs 50/2016 e s.m.i. il Dr./Sig. ((( pratica.rup ))),
           il quale possiede le competenze necessarie a svolgere tale ruolo; 
\item[Art.~2:] di acquistare il bene/servizio ovvero la realizzazione di lavori
        avente le caratteristiche descritte sotto:
\begin{quote}
            ((( pratica.descrizione_acquisto )))
\end{quote}

procedendo all'acquisto tramite affidamento diretto alla Ditta: 

\begin{quote}
((( pratica.nome_fornitore )))\\
((( pratica.ind_fornitore )))
\end{quote}


\item[Art.~3:] che l'importo presunto del lavoro/fornitura/servizio è di 
((( pratica.str_costo_it )))
a carico del/dei fondo/i ((( pratica.stringa_codf ))), ((* if pratica.cup *)) CUP ((( pratica.cup ))), ((* endif *)) Cap. ((( pratica.capitolo ))); 

\item[Art.~4:] di prenotare l'impegno di spesa. 
\end{list}

\vspace{0.5cm}

\begin{flushright}
\begin{minipage}[t]{6cm}
\begin{center}
Il Direttore \\
((( pratica.titolo_direttore ))) ((( pratica.nome_direttore )))
\end{center}
\end{minipage}
\end{flushright}
\end{document}
