%% template: ordine-cnsip.tex. Vers. 1.0

((* include 'header.tex' *))

\begin{document}
\topaddr

\begin{center}
	\textbf{DOCUMENTO DI STIPULA}
\end{center}
\framebox[\linewidth]{%
\begin{minipage}{0.9\linewidth}
In relazione all'Affidamento Diretto di: ((( pratica.descrizione_acquisto )))

CIG: ((( pratica.numero_cig )))((* if pratica.numero_cup *))  \hfill CUP: ((( pratica.numero_cup )))((* endif *))
\end{minipage}
}

\framebox[\linewidth]{ \textbf{Dati generali} }

\textbf{Tipologia di affidamento:} Affidamento diretto \\
\textbf{Criteri di aggiudicazione:} Minor prezzo \\
\textbf{Rischi di interferenza nell'esecuzione della prestazione richiesta:} Nessuno \\

\framebox[\linewidth]{ \textbf{Amministrazione contraente} }

\textbf{Nome Ente:} ((( pratica.sede.sede_it ))) \\
\textbf{Codice Fiscale Ente:} ((( pratica.sede.cod_fisc ))) \\
\textbf{Telefono:} ((( pratica.sede.tel_oss ))) \\
\textbf{Email:} ((( pratica.sede.email_ufficio ))) \\
\textbf{PEC:} ((( pratica.sede.pec_oss ))) \\
\textbf{RUP:} ((( pratica.nome_rup ))) \\
\textbf{Punto Ordinante:} ((( pratica.nome_direttore ))) \\
\textbf{Firmatario del documento di stipula:} ((( pratica.nome_direttore ))) \\

\framebox[\linewidth]{ \textbf{Operatore economico contraente} }

\textbf{Nome\,-\,Denominazione\,-\,Ragione~Sociale:} ((( pratica.fornitore_nome ))) \\
\textbf{Codice Fiscale:} ((( pratica.fornitore_codfisc ))) \\
\textbf{Partita IVA:} ((( pratica.fornitore_partiva ))) \\
\textbf{Sede legale:} ((( pratica.fornitore_sede ))) \\
\textbf{PEC:} ((( pratica.fornitore_pec ))) \\

\framebox[\linewidth]{ \textbf{Oggetto della fornitura} }

((* if pratica.descrizione_ordine *))
((( pratica.descrizione_ordine )))
((* else *))
((( pratica.descrizione_acquisto )))
((* endif *))


\framebox[\linewidth]{ \textbf{Consegna - Verifica della prestazione - Fatturazione} }

\textbf{Termine di esecuzione della prestazione:} ((( pratica.termine_giorni ))) giorni della stipula \\

L'Operatore Economico contraente prende atto di quanto segue.

La verifica finale della conformità della prestazione sarà effettuata
dal Responsabile Unico del Progetto (RUP).

L'INAF provvederà al pagamento del corrispettivo, dietro il previo
invio di regolare fattura da parte dell'aggiudicatario e rilascio di
certificato di regolare esecuzione contrattuale del Responsabile Unico
del Progetto (RUP).

Tutte le fatture dovranno essere trasmesse tramite il Sistema di
Interscambio (SdI) dell'Agenzia delle Entrate, utilizzando il Codice
Univoco Ufficio (CUU): ((( pratica.sede.cuu ))).

Il pagamento del corrispettivo sarà effettuato, entro 30 giorni dal
ricevimento previa verifica della Regolarità  Contributiva,  nel
rispetto di quanto previsto dall'art. 125 del D.lgs. numero 36/2023 e
successive modifiche ed integrazioni, mediante bonifico bancario su conto
corrente dedicato, anche in via non esclusiva, alle commesse pubbliche che
l'Operatore Economico contraente provvederà ad indicare ad INAF entro 7
(sette) giorni dalla data della sua accensione, ove già non ne risulti
intestatario. L'Operatore Economico contraente dovrà inoltre comunicare
entro lo stesso termine le generalità e il codice fiscale delle persone
delegate ad operare su di esso. Ai sensi della legge numero 136/2010 e
successive modifiche ed integrazioni, l'Operatore Economico contraente
\textbf{si obbliga a garantire la tracciabilità dei flussi finanziari relativi al
presente appalto inserendo in fattura i sopra richiamati codici CIG e CUP},
pena la risoluzione del contratto. Il pagamento sarà subordinato alla
verifica d'ufficio della regolarità contributiva dell' Operatore
Economico contraente nonché, alle verifiche previste dall'art. 48
bis del d.P.R. numero 602/1973 e successive modifiche ed integrazioni,
da parte di INAF. L' Operatore Economico contraente si impegna a
comunicare tempestivamente all'INAF le eventuali variazioni delle
coordinate bancarie, esonerando l'INAF in difetto di tale notifica,
da ogni responsabilità per i pagamenti eseguiti, anche ove le predette
variazioni siano pubblicate nei modi di legge.


\textbf{Indirizzo di consegna:} ((( pratica.sede.indirizzo ))). 

\textbf{Dati di fatturazione:} \\
Intestazione: ((( pratica.sede.sede_it ))); \\
Indirizzo: ((( pratica.sede.indirizzo ))); \\
Codice Fiscale: ((( pratica.sede.cod_fisc ))); \\
Partita IVA: ((( pratica.sede.part_iva ))); \\
Codice e-fattura: ((( pratica.sede.cuu ))); \\

\framebox[\linewidth]{ \textbf{Disciplina del contratto} }

\textbf{Articolo 1}

Con l'accettazione del presente documento di stipula l'Operatore
Economico Stipulante ne accetta integralmente le condizioni.

\textbf{Articolo 2}

L'Operatore Economico contraente si obbliga in particolare a:

\begin{enumerate}
	\item eseguire la prestazione oggetto del Contratto, impiegando tutte le
strutture ed il personale necessario per la loro realizzazione secondo
quanto stabilito nel Contratto e negli atti relativi al presente
affidamento;

\item predisporre tutti gli strumenti e le metodologie, comprensivi della
relativa documentazione, atti a garantire elevati livelli di servizio,
ivi compresi quelli relativi alla sicurezza e riservatezza, nonché atti
a consentire a INAF di monitorare la conformità della prestazione alle
norme previste nel Contratto;

\item manlevare e tenere indenne INAF per quanto di rispettiva competenza, dalle
pretese che i terzi dovessero avanzare in relazione ai danni derivanti da
servizi resi in modalità diverse rispetto a quanto previsto nel presente
Contratto, ovvero in relazione a diritti di privativa vantati da terzi;

\item per il personale impiegato a qualsiasi titolo nel presente appalto,
ad ottemperare nei confronti dei propri dipendenti e collaboratori,
a tutti gli obblighi derivanti dalle vigenti disposizioni legislative,
regolamentari e di CCNL di categoria, in materia di retribuzione,
previdenza, assistenza e assicurazione con esclusione di qualsiasi
responsabilità da parte dell'INAF;

\item all'osservanza delle norme e delle disposizioni legislative in materia
di prevenzione dagli infortuni e di igiene sul lavoro, impartendo ai
propri dipendenti e collaboratori precise istruzioni sui rischi specifici
esistenti nell'ambiente di lavoro in cui sono chiamati a prestare la
loro attività. In particolare, l'AFFIDATARIO si impegna a rispettare,
nell'esecuzione delle obbligazioni contrattuali, le disposizioni di cui
al D.lgs. numero 81/2008 e successive modifiche ed integrazioni. 

\item a far osservare, per quanto compatibile, ai propri dipendenti e
collaboratori il Codice di comportamento in materia di anticorruzione del
personale INAF, pubblicato nella sezione “Amministrazione trasparente”
del sito istituzionale dell'INAF. L'Operatore Economico contraente
dichiara di aver preso visione e di essere a conoscenza del contenuto
del predetto Codice; nelle ipotesi di grave violazione delle diposizioni
ivi contenute, la stazione appaltante si riserva la facoltà di risolvere
il contratto;

\item al rispetto delle disposizioni di cui all'art. 53, comma 16-ter del
D.lgs. numero 165/2001 e successive modifiche ed integrazioni in materia
di conferimento di incarichi o contratti di lavoro ad ex dipendenti
della stazione appaltante pena l'obbligo di restituzione dei compensi
illegittimamente percepiti ed accertati ad essi riferiti.
\end{enumerate}

\textbf{Articolo 3}

\begin{enumerate}
	\item Nel caso di inadempienze o di ritardi nell'esecuzione del presente
Contratto l'AFFIDATARIO sarà tenuto al pagamento di una penale pari
a all'uno per mille dell'importo contrattuale, per ogni giorno di
ritardata consegna o inadempimento rispetto al termine indicato nella
diffida.

\item L'applicazione delle penali non preclude all'INAF il diritto di agire
per il risarcimento degli eventuali maggiori danni o per l'eventuale
risoluzione del Contratto.
\end{enumerate}


\textbf{Articolo 4}

L'INAF si riserva il diritto di risolvere il presente Contratto, ai sensi
e per gli effetti dell'art. 1456 del Codice Civile con comunicazione
scritta da inviarsi con raccomandata con avviso di ricevimento (a/r)
o mediante posta elettronica certificata (PEC), con un preavviso di 20
(venti) giorni, nei seguenti casi:

\begin{itemize}
\item[a)] qualora nei confronti dell'appaltatore sia intervenuta l'emanazione
di un provvedimento definitivo che dispone l'applicazione di una o più
misure di prevenzione di cui all'art. 67 e seguenti del D.lgs. numero
159/2011 e successive modifiche ed integrazioni, ovvero sia intervenuta
sentenza di condanna passata in giudicato per frodi nei riguardi della
Stazione Appaltante, di subcontraenti, di subappaltatori, di fornitori,
di lavoratori o di altri soggetti comunque interessati all'appalto,
nonché per violazioni gravi degli obblighi attinenti alla sicurezza
sul lavoro; 

\item[b)] per l'intervenuto accertamento in via definitiva di violazioni gravi
alla normativa previdenziale ed assicurativa, nonché al pagamento di
imposte e tasse. Restano in ogni caso impregiudicati i diritti dell'INAF
al risarcimento di eventuali ulteriori danni e all'incameramento della
garanzia fideiussoria. 

\item[c)] L'INAF si riserva la facoltà di risolvere il contratto nei casi di
cui all'art. 122 del d.l.gs 36/2023.

\item[d)] L'INAF si riserva, inoltre, il diritto di recedere unilateralmente
dal Contratto in qualsiasi momento senza alcun onere a suo carico,
con un preavviso di almeno 20 (venti) giorni solari, da comunicarsi
all'Operatore Economico contraente mediante raccomandata con
avviso di ricevimento (a/r) o mediante posta elettronica certificata
(PEC). In caso di recesso all'Operatore Economico contraente, spetterà
il corrispettivo limitatamente alla prestazione sino ad allora resa,
secondo i corrispettivi e le condizioni previsti nel presente Contratto,
oltre al decimo dell'importo delle forniture non eseguite, in ossequio
a quanto previsto dall'art. 123 del D.Lgs. 36/2023. 
\end{itemize}


\textbf{Articolo 5}

\begin{enumerate}
	\item Con la sottoscrizione del presente atto si assumono tutti gli oneri
assicurativi e previdenziali di legge, nonché l'obbligo di osservare
le norme vigenti in materia di sicurezza sul lavoro e di retribuzione
dei lavoratori dipendenti, nonché si accettano tutte le condizioni
contrattuali e le penalità.

\item Si prende atto che i termini stabiliti nella documentazione allegata
alla procedura di affidamento, relativamente ai tempi di esecuzione del
Contratto, sono da considerarsi termini essenziali ai sensi e per gli
effetti dell'art. 1457 C.C.

\item Il presente Documento di Stipula è esente da registrazione ai sensi de
Testo Unico del 22/12/1986 n. 917, art. 6 e s.m.i., salvo che in caso
d'uso ovvero da quanto diversamente e preventivamente esplicitato dalla
Stazione Appaltante nelle Condizioni Particolari di Fornitura.

\item Sono a carico dell'Operatore Economico contraente tutti gli eventuali
oneri tributari, le eventuali spese contrattuali ivi comprese quelle
relative all'eventuale assoggettamento ad imposta di bollo.
\end{enumerate}

\textbf{Articolo 6}

\begin{enumerate}
	\item Il presente documento si intende privo di effetti ove privo di
sottoscrizione con valida firma digitale.

\item L'Operatore Economico contraente si impegna a ritrasmettere copia
del presente atto controfirmata con firma digitale del legale
rappresentante all'indirizzo PEC della stazione appaltante:
 inafoaarcetri@pcert.postecert.it.

\item Il rapporto oggetto della presente stipulazione si intenderà perfezionato
solo al momento in cui la medesima Stazione Appaltante avrà notizia
della sottoscrizione del presente atto da parte dell'Operatore Economico
contraente, come sopra specificato.
\end{enumerate}

((* if pratica.ord_firma_vicario *))
((* include 'vicario.tex' *))
((* else *))
((* include 'direttore.tex' *))
((* endif *))
\end{document}
