%% Template: rdo-ub143k.tex. Version 1.0

((* include 'header.tex' *))

((* if provvisorio *))
\watermark{\put(150,-300){\rotatebox{45}{\Huge ((( provvisorio ))) }}}
((* endif *))


\begin{document}
\topaddr

((* if debug *))
\begin{quotation}
	\textbf{DEBUG:} template = \texttt{rdo-ub143k.tex}
\end{quotation}
((* endif *))

\begin{center}
\textbf{RICHIESTA DI OFFERTA}
\end{center}

~\\
\textbf{Affidamento diretto della fornitura di ((( pratica.descrizione_acquisto )))}

\textbf{CIG: ((( pratica.numero_cig )))
((* if pratica.numero_cup *)) - CUP: ((( pratica.numero_cup ))) ((* endif *))}

\textbf{OGGETTO: Affidamento Diretto} di cui in epigrafe, CIG: ((( pratica.numero_cig )))

\textbf{STAZIONE APPALTANTE: Istituto Nazionale di Astrofisica (INAF) - Osservatorio Astrofisico di Arcetri}

\textbf{DISCIPLINA DELLA PROCEDURA DI SELEZIONE}

La procedura è regolata dalla presente Richiesta, dal Patto di integrità
e per quanto occorrer possa dagli altri allegati alla presente richiesta
di offerta oltre che dal Decreto Legislativo n. 36/2023.


\textbf{RESPONSABILE UNICO DEL PROGETTO (RUP)}

Responsabile Unico del progetto (RUP), ai sensi dell'art. 15 del
Decreto Legislativo 31 marzo 2023, numero 36, è
 ((( pratica.nome_rup ))), Largo Enrico Fermi n. 5 - 50130 Firenze
 (FI), tel. 055-2752-(((pratica.interno_rup))),
e-mail: ((( pratica.email_rup ))), PEC inafoaarcetri@pcert.postecert.it


\textbf{OGGETTO DELL'AFFIDAMENTO}

L'affidamento ha ad oggetto la fornitura di  ((( pratica.descrizione_acquisto ))).

Si chiede in particolare di confermare o migliorare il preventivo
rilasciato in sede di preliminare esplorazione di mercato: 
(Cfr. Vs. Offerta Nr ((( pratica.numero_offerta ))) del ((( pratica.data_offerta ))).

\textbf{IMPORTO A BASE DELL'OFFERTA}

\textbf{Euro} ((( pratica.costo_base ))), IVA esclusa.

\textbf{REVISIONE DEI PREZZI}

Posto che l'affidamento in oggetto si sostanzia in una fornitura priva
dei caratteri della ricorrenza o della periodicità delle prestazioni
richieste, da eseguire in un arco di tempo estremamente limitato, tanto da
escludere che essa possa venire condizionata da fluttuazioni di mercato,
si ritiene sostanzialmente inapplicabile al caso di specie il meccanismo
della revisione dei prezzi di cui all'art. 60 Decreto Legislativo
31 marzo 2023 n. 36, alla cui disciplina, per quanto occorrer possa,
in questa sede comunque si rinvia.

\textbf{GARANZIA DEFINITIVA}

Si ricorda che, in ossequio all'art. 53, comma 4, Decreto Legislativo
31 marzo 2023 n. 36, è dovuta la prestazione di garanzia definitiva per
l'esecuzione dei contratti, pari al 5\% (cinque per cento) dell'importo
contrattuale, da prestare entro e non oltre il giorno precedente a quello
della stipula del contratto, comunicato all'affidatario.

\ul{Resta altres\`i (ed in ogni caso) insindacabile facolt\`a della
Stazione Appaltante di esentare altrimenti l'Operatore Economico dalla
prestazione della garanzia definitiva, in ossequio a quanto disposto
dal gi\`a citato art. 53, comma 4 D. Lgs. 36/2023.}

\textbf{REQUISITI GENERALI}

Costituisce causa di esclusione la sussistenza delle cause di cui
all'art. 94 e dell'art. 95 del Decreto Legislativo 31 marzo 2023,
numero 36. Sono comunque esclusi gli operatori economici che abbiano
affidato incarichi in violazione dell'art. 53, comma 16-ter, del
Decreto Legislativo 30 marzo 2001, numero 165.

Costituisce causa di esclusione il mancato rispetto, al momento della
presentazione dell'offerta, degli obblighi in materia di lavoro delle
persone con disabilità di cui alla legge 12 marzo 1999, numero 68,
oltre che ai sensi dell'art. 80, comma 5, lettera i), del Decreto
Legislativo 18 aprile 2016, numero 50.

È comunque escluso l'operatore economico che abbia affidato incarichi
in violazione dell'articolo 53, comma 16-ter, del Decreto Legislativo
30 marzo 2001, numero 165 a soggetti che hanno esercitato, in qualità
di dipendenti, poteri autoritativi o negoziali presso l'amministrazione
affidante negli ultimi tre anni.

La mancata accettazione delle clausole contenute nel protocollo
di legalità/patto di integrità e il mancato rispetto dello stesso
costituiscono causa di esclusione dalla gara, ai sensi dell'articolo
83 bis del Decreto legislativo 6 settembre 2011, numero 159.

\textbf{VERIFICA DEI REQUISITI}

Trattandosi di affidamento di valore economico stimato inferiore
ad Euro 40.000,00, la Stazione Appaltante si riserva la facoltà
di procedere alla verifica dei requisiti, mediante dichiarazione
sostitutiva resa dall'Operatore Economico su form messo a disposizione
dall'Amministrazione. In tale caso la Stazione Appaltante si riserva
in ogni caso il diritto di procedere, anche a campione, alla verifica
delle dichiarazioni rese.

\textbf{SUBAPPALTO}

Ai sensi dell'articolo 119 del D.lgs. numero 36/2023 e successive
modifiche ed integrazioni,
\ul{l'affidatario eseguir\`a in proprio i servizi compresi nel
contratto}. Il subappalto è consentito negli esclusivi limiti e con
l'osservanza delle prescrizioni disposti dalla norma citata.

Il concorrente indica all'atto dell'offerta le parti del
servizio/fornitura che intende subappaltare. In caso di mancata
indicazione delle parti da subappaltare il subappalto è vietato. 

L'affidatario e il subappaltatore sono responsabili in solido nei
confronti della stazione appaltante dell'esecuzione delle prestazioni
oggetto del contratto di subappalto.

Ai sensi del comma 3 lett. a) dell'art. 119 del D.lgs. numero 36/2023,
non si configurano come attività affidate in subappalto l'affidamento
di attività secondarie, accessorie o sussidiarie a lavoratori autonomi,
per le quali occorre effettuare comunicazioni alla stazione appaltante.

\textbf{MODALITÀ DI PRESENTAZIONE DELL'OFFERTA}

L'operatore economico dovrà inviare la propria offerta migliorativa
o confermativa di quella già resa in sede di indagine informale
di mercato, mediante formale proposizione della stessa, da inviare
avvalendosi dello specifico canale di trasmissione sulla piattaforma
di \textit{e-procurement"} dell'Istitituto Nazionale di Astrofisica
denominata "UBUY", su carta propria intestata (su documento salvato
in formato PDF/A e firmato digitalmente dal Legale Rappresentante del
Soggetto Giuridico offerente), entro il giorno \textbf{((( pratica.fine_gara )))} alle ore
\textbf{23:59} con indicazione, pena l'esclusione, del domicilio
eletto e \textbf{nell'oggetto} della dicitura: \textbf{AFFIDAMENTO
DIRETTO FORNITURA DI (((pratica.descrizione_acquisto))),} \textbf{CIG:}
((( pratica.numero_cig )))%
((* if pratica.numero_cup *)) \textbf{ - CUP:} ((( pratica.numero_cup )))((* endif *)),
unitamente alla
seguente documentazione (allegata alla richiesta di conferma di offerta)
da completare e salvare in formato "PDF/A", firmata digitalmente dal
Legale Rappresentante del Soggetto Giuridico offerente o dal Procuratore
dello stesso (se diverso - \ul{in quest'ultimo caso dovr\`a essere
allegata anche la copia conforme all'originale della procura}):

\begin{itemize}
\item Presente Richiesta di Offerta \ul{controfirmata per presa visione e accettazione dall'offerente};
\item Dichiarazioni amministrative;
\item Documento di Gara Unico Europeo (DGUE).
\end{itemize}


L'offerta redatta su carta intestata dovrà indicare in cifre e in lettere il prezzo offerto, al
netto dell'IVA. In caso di discordanza tra l'indicazione in cifre
e quella in lettere sarà ritenuta valida quella in lettere.

\textbf{SPESE DI TRASPORTO}: L'offerta dovrà essere redatta, nel
rispetto dell'importo posto a base della procedura (vedi sopra),
\ul{comprensiva di eventuali spese di trasporto}.

L'offerta dovrà riportare chiaramente i tempi di consegna dalla
ricezione dell'ordine.

L'offerta dovrà avere una validità di 180 (centoottanta) giorni
dalla data di scadenza fissata per la ricezione delle offerte.

\textbf{DOCUMENTO DI GARA UNICO EUROPEO (DGUE/ESPD)}

Il concorrente predispone il Documento di Gara Unico
Europeo (DGUE/ESPD) procedendo alla redazione dello stesso,
mediante compilazione e download del modulo sulla piattaforma
UE\footnote{https://espd.eop.bg/espd-web/filter?lang=it}, previa importazione del
file in formato ".xml", (allegato alla presente richiesta di offerta
unitamente alle istruzioni di importazione, compilazione e download) -
avvalendosi di questa opzione il modulo potrà indifferentemente essere
redatto in lingua italiana o in lingua inglese;

Il Documento di Gara Unico Europeo (ESPD/ESPD) \textbf{\ul{dovr\`a
essere da ultimo salvato in formato nativo "PDF/A" e sottoscritto
digitalmente dal rappresentante legale dell'Operatore Economico
partecipante alla procedura di selezione}}.

L'Operatore Economico deve rendere tutte le informazioni richieste
nel DGUE mediante la compilazione delle parti pertinenti.

\textbf{STIPULAZIONE DEL CONTRATTO}

La stipulazione del contratto avverrà a seguito dell'approvazione
dell'offerta mediante sottoscrizione del documento di stipula.

L'INAF - Osservatorio Astrofisico di Arcetri potrà decidere di non
procedere alla contrattualizzazione se l'offerta non risulti conveniente
o idonea in relazione all'oggetto del contratto, senza che al riguardo
il Soggetto Giuridico offerente possa avanzare alcuna pretesa.

La contrattualizzazione è subordinata all'accertamento dell'assenza
delle cause di esclusione previste dagli artt. 94 e 95 del Decreto
Legislativo 31 marzo 2023, numero 36 e all'esito positivo della verifica
del possesso dei requisiti prescritti dalla presente lettera di invito.

Sono a carico dell'aggiudicatario, ove previste, tutte le spese
contrattuali, gli oneri fiscali quali imposte e tasse - ivi comprese
quelle di registro ove dovute - relative alla stipulazione del contratto.

\textbf{TRACCIABILITÀ DEI FLUSSI FINANZIARI}

Il contratto d'appalto è soggetto agli obblighi in tema di
tracciabilità dei flussi finanziari di cui alla Legge 13 agosto 2010,
numero 136. L'affidatario deve comunicare alla stazione appaltante:

\begin{itemize}

\item gli estremi identificativi dei conti correnti bancari o postali
dedicati, con l'indicazione dell'opera/servizio/fornitura alla quale
sono dedicati;

\item le generalità e il codice fiscale delle persone delegate ad
operare sugli stessi;

\item ogni modifica relativa ai dati trasmessi. 
\end{itemize}

La comunicazione deve essere effettuata entro sette giorni dall'accensione
del conto corrente ovvero, nel caso di conti correnti già esistenti,
dalla loro prima utilizzazione in operazioni finanziarie relative ad
una commessa pubblica. In caso di persone giuridiche, la comunicazione
de quo deve essere sottoscritta da un legale rappresentante ovvero da
un soggetto munito di apposita procura. L'omessa, tardiva o incompleta
comunicazione degli elementi informativi comporta, a carico del soggetto
inadempiente, l'applicazione di una sanzione amministrativa pecuniaria
da 500 a 3.000 euro.

Il mancato adempimento agli obblighi previsti per la tracciabilità
dei flussi finanziari relativi all'appalto comporta la risoluzione di
diritto del contratto. In occasione di ogni pagamento all'appaltatore
o di interventi di controllo ulteriori si procede alla verifica
dell'assolvimento degli obblighi relativi alla tracciabilità dei
flussi finanziari.

Il contratto è sottoposto alla condizione risolutiva in tutti i casi
in cui le transazioni siano state eseguite senza avvalersi di banche
o di Società Poste Italiane S.p.a. o anche senza strumenti diversi
dal bonifico bancario o postale che siano idonei a garantire la piena
tracciabilità delle operazioni per il corrispettivo dovuto in dipendenza
del presente contratto.

\textbf{LINGUA DELLA PROCEDURA DI AFFIDAMENTO}

La \textbf{lingua ufficiale} della presente procedura di affidamento è unicamente quella \textbf{italiana}. 

L'\textbf{offerta ("quotation")} potrà eccezionalmente essere
redatta in \textbf{lingua inglese}. Ogni altro documento/allegato
(fatta eccezione per certificazioni/estratti di verifica requisiti
rese da operatore straniero) dovrà essere compilato e depositato in
\textbf{lingua italiana}.

\textbf{TRATTAMENTO DEI DATI PERSONALI}

I dati personali saranno raccolti e trattati conformemente al "Regolamento
del Parlamento e del Consiglio Europeo del 27 aprile 2016, numero UE
2016/679,  relativo alla protezione delle persone fisiche con riguardo
al trattamento dei dati personali, nonché alla libera circolazione
di tali dati, che abroga la Direttiva dell'Unione Europea del 24
ottobre 1995, numero 95/46/CE", denominato anche "Regolamento Generale
sulla Protezione dei Dati" (GDPR) e dalla normativa italiana vigente,
esclusivamente ai fini del presente procedimento e secondo quanto indicato
nell'informativa disponibile presso la seguente pagina web:

http://www.inaf.it/it/privacy/informative/informativa-fornitori-beni-e-servizi/view

L'Ente raccoglie le seguenti categorie di dati richiesti per la
presente procedura, in base alla normativa in materia di appalti
e contrattualistica pubblica, per valutare il possesso dei requisiti
e delle qualità richiesti per la partecipazione alla procedura nel cui
ambito i dati stessi sono acquisiti; pertanto, la loro mancata indicazione
può precludere l'effettuazione della relativa istruttoria.

L'esercizio dei diritti di cui agli articoli 15 e ss. del Regolamento UE
2016/679 potrà essere fatto valere, senza alcuna formalità, rivolgendo
apposita istanza al Responsabile della Protezione dei Dati dell'Istituto
(email: rpd@inaf.it) Il Titolare del trattamento è l'Istituto Nazionale
di Astrofisica con sede legale in Viale del Parco Mellini, 84 - 00136
Roma, Italia.

((* include 'rup.tex' *))
\end{document}
