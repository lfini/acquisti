\documentclass[a4paper,12pt]{letter}
\usepackage[T1]{fontenc} 
\usepackage[utf8]{inputenc}
\usepackage[italian]{babel} 
\usepackage{eurosym}

%% Template: autorizzazione.tex. Version 1.0

((* include 'header.tex' *))

\begin{document}
\topaddr


\begin{center}
\textbf{PROGETTO DI FORNITURA \\ (RICHIESTA DI ACQUISIZIONE DI BENI)}
\end{center}

~\\
\textbf{Soggetto richiedente: ((( pratica.nome_responsabile )))} \\

In qualità di:  \textit{vedi e-mail del 17/3}

~\\
\begin{center}
\textbf{\underline{Richiede l'acquisizione del seguente bene}}
\end{center}

~\\
\textbf{Descrizione tecnica del bene/servizio richiesto:}
\begin{quote}
((( pratica.descrizione_acquisto )))
\end{quote}

~\\
Cfr. preventivo Operatore Economico 

~\\
\textbf{Motivazioni relative alla necessità dell’acquisto: }
\begin{quote}
((( pratica.motivazione_acquisto )))
\end{quote}


\textbf{PROCEDURA: }

Affidamento diretto ai sensi del combinato disposto dagli articoli 17 comma 2 e 50, comma 1 lett. b) D.Lgs. 36/2023.\\
L’esplorazione informale di mercato sin qui condotta ha consentito infatti di individuare:
\begin{itemize}
\item un importo stimato del progettato affidamento sicuramente al di sotto della soglia contemplata dalla norma innanzi richiamata;
\item l’assenza di una Convenzione in ambito Consip;
\item presenza su MePA di almeno una ditta di riferimento che assicurasse le condizioni (tecniche e di budget) ritenute ottimali;
\item una ditta in grado di rispondere efficacemente al quadro esigenziale connesso al presente affidamento,
individuata sul mercato all’esito di una informale esplorazione e specificamente:
((( pratica.nome_fornitore ))) con sede legale in ((( pratica.ind_fornitore )))

		\textit{codice fiscale C.F. e partita IVA P.Iva: vedi mail del 17/3}

con la quale avviare una procedura di affidamento su piattaforma “MePA”.
\end{itemize}

Si richiama inoltre l’art. 10, comma 3, D. Lgs. 218/2016 a mente
del quale “Le disposizioni di cui all'articolo 1, commi 450, primo
periodo, e 452, primo periodo, della legge 27 dicembre 2006, n. 296, non
si applicano agli Enti per l'acquisto di beni e servizi funzionalmente
destinati all'attività di ricerca”.

\textbf{VERIFICA DELL’INTERESSE TRANSFRONTALIERO}

Si ritiene peraltro l’insussistenza, in relazione all’affidamento in
oggetto, dei presupposti giustificanti l’applicazione dell’art. 48,
comma 2, Decreto Legislativo 31 marzo 2023 n. 36, in considerazione delle
specificità dell’affidamento, che non consentono di identificare in
esso un interesse transfrontaliero certo. 

\textbf{VERIFICA DEL RISPETTO DEL PRINCIPIO DI ROTAZIONE DEGLI AFFIDAMENTI}

In relazione all’affidamento in oggetto è verificato il rispetto del
principio di rotazione degli affidamenti, di cui all’articolo 49 del
Decreto Legislativo 31 marzo 2023 n. 36. 

\textbf{L’acquisto non prevede ripetizioni o altre opzioni.}

Ai fini della inventariazione, ove necessaria in ragione della fornitura
in oggetto, è riservata la successiva precisazione degli effettivi
utilizzatori e della effettiva localizzazione dei beni da inventariare.

\textbf{La consegna della fornitura deve avvenire presso:}

((( pratica.sede.sede_it ))). ((( pratica.sede.indirizzo )))
\par\noindent\rule{\textwidth}{0.4pt}

\textbf{Spesa complessiva stimata (IVA esclusa): }

Oggetto dell'affidamento: ((( pratica.costo.importo ))) (((pratica.costo.valuta )))

Costi di trasporto: ((( pratica.costo_trasporto )))  (((pratica.costo.valuta )))

\textit{IVA: vedi e.mail 17/3}

\textit{TOTALE: valore Euro: vedi e-mail 17/3}

A gravare su: Ob. Fu ((( pratica.stringa_codf ))) - 
Responsabile Fondi: ((( pratica.nome_responsabile )))
\vspace{0.5cm}

\begin{flushright}
\begin{minipage}[c]{6cm}
\begin{center}
Il Richiedente

((( pratica.nome_responsabile )))

\end{center}
\end{minipage}
\end{flushright}

\end{document}
