%% Template: decisione-inf5k.tex. Version 1.0

((* include 'header.tex' *))

((* if provvisorio *))
\watermark{\put(150,-300){\rotatebox{45}{\Huge ((( provvisorio ))) }}}
((* endif *))

\begin{document}
\topaddr

((* if debug *))
\begin{quotation}
	\textbf{DEBUG:} template = \texttt{decisione-inf5k.tex}
\end{quotation}
((* endif *))

Pratica N. ((( pratica.numero_pratica )))

\begin{flushright}
Decisione N. ((( pratica.numero_decisione )))
\end{flushright}

\begin{center}
\framebox[\linewidth]{%
\begin{minipage}{0.9\linewidth}
Decisione di contrarre del ((( pratica.data_decisione ))) – Affidamento diretto
	della fornitura di ((( pratica.descrizione_acquisto )))
CIG: ((( pratica.numero_cig ))) - CUP: ((( pratica.numero_cup )))
\end{minipage}
}

Il Direttore

\end{center}
\textbf{VISTA~}	la Legge 7 agosto 1990, numero 241, e successive modifiche
ed integrazioni, che contiene \textbf{"Nuove norme in materia di procedimento
amministrativo e di diritto di accesso ai documenti amministrativi"},
ed in particolare gli articoli 4, 5 e 6;

\textbf{VISTO~}	il Decreto Legislativo 30 marzo 2001, numero 165, e
successive modificazioni ed integrazioni, che contiene \textbf{"Norme generali
sull’ordinamento del lavoro alle dipendenze delle amministrazioni
pubbliche"} ed in particolare gli articoli 1, 2, 4, 16 e 17;

\textbf{VISTO~}	il Decreto del Presidente della Repubblica 27 febbraio
2003, numero 97, con il quale è stato emanato il \textbf{"Regolamento concernente
l’amministrazione e la contabilità degli enti pubblici di cui alla
Legge 20 marzo 1975, n. 70"}, ed in particolare gli articoli 30, 31 e 32;

\textbf{VISTO~}	il Decreto Legislativo del 23 luglio 1999, numero 296,
pubblicato nella Gazzetta Ufficiale della Repubblica Italiana, Serie
Generale, del 26 agosto 1999, n. 200, che istituisce lo \textit{"Istituto
Nazionale di Astrofisica"} (INAF);

\textbf{VISTO~}	il Decreto Legislativo 4 giugno 2003, numero 138,
pubblicato nella Gazzetta Ufficiale della Repubblica Italiana, Serie
Generale, del 19 giugno 2003, numero 140, che disciplina il \textbf{"Riordino
dello Istituto Nazionale di Astrofisica"}, come modificato e integrato
dallo "Allegato 2" del Decreto Legislativo 21 gennaio 2004, numero 38,
che, tra l’altro, istituisce, ai sensi dell’articolo 1 della Legge
6 luglio 2002, n. 137, lo \textbf{"Istituto Nazionale di Ricerca Metrologica
(INRIM)"};

\textbf{VISTO~}	il Decreto Legislativo 30 giugno 2003, numero 196, con
il quale è stato adottato il \textbf{"Codice in materia di protezione dei dati
personali"};

\textbf{VISTO~}	il "Regolamento (UE) 2016/679 del Parlamento e del
Consiglio Europeo del 27 aprile 2016, relativo alla protezione delle
persone fisiche con riguardo al trattamento dei dati personali, nonché
alla libera circolazione di tali dati, che abroga la Direttiva 95/46/CE",
denominato anche \textbf{"Regolamento Generale sulla Protezione dei Dati"
("RGPD")}, in vigore dal 24 maggio 2016 e applicabile a decorrere dal 25
maggio 2018;

\textbf{VISTO~}	il Decreto Legislativo 10 agosto 2018, numero 101,
che contiene alcune Disposizioni per l’adeguamento della normativa
nazionale alle disposizioni del Regolamento (UE) 2016/679 del Parlamento
e del Consiglio Europeo del 27 aprile 2016, relativo alla protezione
delle persone fisiche con riguardo al trattamento dei dati personali,
nonché alla libera circolazione di tali dati, che abroga la Direttiva
95/46/CE", denominato anche \textbf{"Regolamento Generale sulla Protezione dei
Dati"} ("RGPD");

\textbf{VISTO~}	il Decreto Legge 6 luglio 2011, numero 98, che contiene
\textbf{"Disposizioni urgenti per la stabilizzazione finanziaria"}, convertito,
con modificazioni, dalla Legge 15 luglio 2011, numero 111, ed, in
particolare, l’articolo 11, che disciplina gli \textbf{"Interventi per la
razionalizzazione dei processi di approvvigionamento di beni e servizi
della Pubblica Amministrazione"}, e che dispone, tra l’altro, che,
qualora \textit{"... non si ricorra alle convenzioni di cui all’articolo
1, comma 449, della Legge 27 dicembre 2006, numero 296, gli atti e i
contratti posti in essere in violazioni delle disposizioni sui parametri
contenuti nell’articolo 26, comma 3, della Legge 23 dicembre 1999,
numero 488, sono nulli e costituiscono illecito disciplinare e determinano
responsabilità erariale ...}

\textbf{VISTO~}	il Decreto Legge 7 maggio 2012, numero 52, che contiene
"Disposizioni urgenti per la razionalizzazione della spesa pubblica",
convertito, con modificazioni, dalla Legge 6 luglio 2012, numero 94, ed,
in particolare, l’articolo 7, che ha modificato l’articolo 1, commi
449 e 450, della Legge del 27 dicembre 2006, numero 296, e modificato
negli importi dalla Legge 30 dicembre 2018, articolo 1, comma 130,
prevedendo, tra l’altro, che:

\begin{itemize}
    
\item nel rispetto del \textit{"... sistema delle convenzioni di cui
agli articoli 26 della Legge 23 dicembre 1999, numero 488 e successive
modificazioni, e 58 della Legge 23 dicembre 2000, numero 388, tutte
le amministrazioni statali centrali e periferiche, ivi compresi gli
istituti e le scuole di ogni ordine e grado, le istituzioni educative e
le istituzioni universitarie, nonché gli enti nazionali di previdenza
e assistenza sociale pubblici e le agenzie fiscali di cui al Decreto
Legislativo 30 luglio 1999, numero 300, sono tenute ad approvigionarsi
utilizzando le Convenzioni Quadro..." stipulate dalla \textbf{Concessionaria
dei Sistemi Informativi Pubblici (CONSIP)} le "... amministrazioni
statali centrali e periferiche, ad esclusione degli istituti e delle
scuole di ogni ordine e grado, delle istituzioni educative e delle
istituzioni universitarie, nonché gli enti nazionali di previdenza e
di assistenza sociale pubblici e le agenzie fiscali di cui al Decreto
Legislativo 30 luglio 1999, n. 300, \textbf{per gli acquisti di beni e servizi
di importo pari o superiore a 5.000 euro e al di sotto della soglia di
rilievo comunitario, sono tenute a fare ricorso al "Mercato Elettronico
della Pubblica Amministrazione"} di cui all’articolo 328, comma 1,
del Regolamento emanato con Decreto del Presidente della Repubblica 5
ottobre 2010, numero 207..."};

\item fermi restando "...gli obblighi e le facoltà previsti al
comma 449 del presente articolo, le altre amministrazioni pubbliche di
cui all’articolo 1 del Decreto Legislativo 30 marzo 2001, numero 165,
nonché le autorità indipendenti, \textbf{per gli acquisti di beni e servizi di
importo pari o superiore a 5.000 euro e inferiore alla soglia di rilievo
comunitario sono tenute a fare ricorso al "Mercato Elettronico della
Pubblica Amministrazione" ovvero ad altri mercati elettronici istituiti
ai sensi del medesimo articolo 328 ovvero al sistema telematico messo a
disposizione dalla centrale regionale di riferimento per lo svolgimento
delle relative procedure ..."} ;

\end{itemize}

\textbf{VISTO~}	il Regolamento sull’Amministrazione, sulla Contabilità
e sull’Attività Contrattuale dell’INAF, pubblicato sul S.O. n. 185
alla G.U. Serie Generale n. 300 del 23 dicembre 2004, in particolare
nel suo articolo 14 come modificato al comma 4 con Delibera n. 100
del 8 novembre 2005 pubblicata sulla G.U. n. 31 serie generale del
7 febbraio 2006 e con Delibera n. 46 del 2 luglio 2009, approvata dal
Ministero dell’Istruzione, dell’Università e della Ricerca con nota
prot. n. 628 del 29 luglio 2009;

\textbf{VISTO~}	il Decreto Legislativo 25 novembre 2016 n. 218
"Semplificazione delle attività degli enti pubblici di ricerca ai
sensi dell’articolo 13 della legge 7 agosto 2015, n. 124" ed in
particolare l’Art. 10;

\textbf{VISTO~}	lo Statuto dell’INAF, adottato dal Consiglio di
Amministrazione con delibera n. 42 del 25 maggio 2018 ed entrato in
vigore il 24 settembre 2018 e successive modifiche ed integrazioni;

\textbf{VISTO~}	 il Regolamento di Organizzazione e Funzionamento
dell’INAF (ROF), approvato dal Consiglio di Amministrazione con delibera
del 5 giugno 2020 n. 46, modificato dal medesimo Organo di Governo con
Delibera del 29 aprile 2021, numero 21, pubblicato in data 24 giugno
2021 ed entrato in vigore il 9 luglio 2021;

\textbf{VISTO~}	il vigente "Piano Triennale per la Informatica nella Pubblica
Amministrazione";

\textbf{VISTA~}	la Direttiva 2014/24/UE del Parlamento Europeo e del
Consiglio, del 26 febbraio 2014, sugli appalti pubblici e che abroga la
direttiva 2004/18/CE;  

\textbf{VISTA~}	la Direttiva 2014/25/UE del Parlamento Europeo e del
Consiglio, del 26 febbraio 2014, sulle procedure d'appalto degli enti
erogatori nei settori dell'acqua, dell'energia, dei trasporti e dei
servizi postali e che abroga la direttiva 2004/17/CE; 

\textbf{VISTO~}	il Decreto Legislativo 31 marzo 2023, numero 36, con il
quale:
\begin{itemize}

\item  è stata data piena attuazione alla Legge 21 giugno 2022,
numero 78 "Delega al Governo in materia di contratti pubblici";

\item è stato adottato il nuovo \textbf{"Codice dei Contratti Pubblici"},
    pubblicato nel Supplemento Ordinario numero 12 alla Gazzetta Ufficiale
    della Repubblica Italiana, Serie Generale, del 31 marzo 2023, numero 77
\end{itemize}

\textbf{VISTI~}	gli articoli da 19 a 36 del D. Lgs. 31 marzo 2023 n. 36,
relativa alla Parte II "Della digitalizzazione del ciclo di vita dei
contratti".

\textbf{VISTO~}                   l’articolo 14 del Decreto Legislativo
31 marzo 2023, numero 36, "Soglie di rilevanza    europea e metodo di
calcolo dell’importo stimato degli appalti. Disciplina dei contratti
misti"; 

\textbf{VISTO~}	l’articolo 49 del Decreto Legislativo 31 marzo 2023,
numero 36, rubricato "Principio di rotazione degli affidamenti";


\textbf{VISTO~}	l’articolo 50 del Decreto Legislativo 31 marzo 2023,
numero 36, rubricato "Procedure per l’affidamento" che prevede,
tra l’altro, che le stazioni appaltanti "…procedono all'affidamento
dei contratti di lavori, servizi e forniture di importo inferiore alle
soglie di cui all'articolo 14 con le seguenti modalità:


\begin{enumerate}

\item[a)]  \textbf{affidamento diretto per i lavori di importo inferiore a 150.000
euro, anche senza consultazione di più operatori economici}, assicurando
che siano scelti soggetti in possesso di documentate esperienze pregresse
idonee all'esecuzione delle prestazioni contrattuali, individuati anche
tra gli iscritti in elenchi o albi istituiti dalla stazione appaltante;

\item[b)]  \textbf{affidamento diretto dei servizi e delle forniture, ivi compresi
i servizi di ingegneria e architettura} e l'attività di progettazione,
di importo inferiore a \textbf{140.000 euro, anche senza consultazione di più
operatori economici}, assicurando che siano scelti soggetti in possesso di
documentate esperienze pregresse idonee all'esecuzione delle prestazioni
contrattuali, individuati anche tra gli iscritti in elenchi o albi
istituiti dalla stazione appaltante;

\item[c)] \textbf{procedura negoziata senza bando, previa consultazione di almeno
cinque operatori economici, ove esistenti}, individuati in base a indagini
di mercato o tramite elenchi di operatori economici, per i lavori \textbf{di
importo pari o superiore a 150.000 euro e inferiore a 1 milione di euro};

\item[d)] \textbf{procedura negoziata senza bando, previa consultazione di almeno
dieci operatori economici, ove esistenti}, individuati in base a indagini
di mercato o tramite elenchi di operatori economici, per i \textbf{lavori di
importo pari o superiore a 1 milione di euro e fino alle soglie di cui
all'articolo 14}, fatta salva la possibilità di ricorrere alle procedure
di scelta del contraente di cui alla Parte IV del presente Libro;

\item[e)] \textbf{procedura negoziata senza bando, previa consultazione di almeno
cinque operatori economici}, ove esistenti, individuati in base ad indagini
di mercato o tramite elenchi di operatori economici, per \textbf{l'affidamento di
servizi e forniture, ivi compresi i servizi di ingegneria e architettura
e l'attività di progettazione, di importo pari o superiore a 140.000
euro} e fino alle soglie di cui all'articolo 14…";

\end{enumerate}

\textbf{VISTO~} il Regolamento Delegato (UE)
2023/2494 della Commissione del 15 Novembre 2023 che modifica la direttiva
2014/24/UE del Parlamento europeo e del Consiglio per quanto riguarda
le soglie degli appalti pubblici di forniture, servizi e lavori e dei
concorsi di progettazione;

\textbf{VISTO~}  in particolare l’articolo 1 del Regolamento Delegato (UE)
2023/2494 che stabilisce: "La direttiva 2014/24/UE è così modificata: 

\begin{enumerate}
\item l’articolo 4 è così modificato: 
\begin{enumerate}
\item[a)]  alla lettera a), «5.382.000 EUR» è sostituito da «5.538.000 EUR»;
\item[b)]  alla lettera b), «140.000 EUR» è sostituito da «143.000 EUR»; 
\item[c)]  alla lettera c), «215.000 EUR» è sostituito da «221.000 EUR»; 
\end{enumerate}

\item all’articolo 13, il primo comma è così modificato: 
\begin{enumerate}
\item[a)]  alla lettera a), «5.382.000 EUR» è sostituito da «5.538.000 EUR»; 
\item[b)]  alla lettera b), «215.000 EUR» è sostituito da «221.000 EUR».
\end{enumerate}
\end{enumerate}

\textbf{VISTO~} l’articolo 15 del Decreto Legislativo 31
marzo 2023, numero 36, "Responsabile unico del progetto (RUP)" e
l’allegato I.2 al medesimo Decreto "Attività del RUP";

\textbf{CONSIDERATO~} che in ordine all’affidamento in oggetto
non si è riscontrata la sussistenza di convenzioni o accordo quadro
stipulati da Consip tutt’ora in vigore o comunque economicamente
preferibili all’autonomo ricorso al mercato;

\textbf{CONSIDERATO~} altresì che, in ordine
all’affidamento in oggetto, sulla piattaforma del Mercato Elettronico
della Pubblica Amministrazione (MePA) non è presente alcun fornitore
capace di rispondere in maniera puntuale alle specificità dell’oggetto
dell’affidamento;

\textbf{RILEVATA~} infine l’assenza anche sulla piattaforma e-procurement per la
gestione delle gare dematerializzate denominato "UBUY"
(\texttt{http://inaf.ubuy.cineca.it/PortaleAppalti/it/homepage.wp}) in uso a
INAF – Istituto Nazionale di Astrofisica – di Operatore Economico in
grado di soddisfare le richieste legate all’oggetto dell’affidamento;

\textbf{VISTO~} l’esito della preliminare
esplorazione di mercato che ha condotto alla individuazione come migliore
Operatore Economico in grado di assolvere all’esecuzione dell’appalto
in oggetto ((( pratica.nome_ditta ))) con sede in ((( pratica.indirizzo_ditta )))
codice fiscale ((( pratica.codfisc_ditta ))) e partita IVA ((( pratica. partiva_ditta ))),
in quanto:

\begin{itemize}
\item  le prestazioni dell’oggetto fornito dall’Operatore Economico
appaiono le più idonee alla realizzazione del Progetto;

\item 
l’offerta da esso presentata si è rilevata plausibile e conforme
ai valori di mercato;

\item tale O.E. vanta una solida e riconosciuta
competenza nel settore di riferimento;

\end{itemize}

\textbf{RITENUTA~}  l’insussistenza, in relazione
all’affidamento in oggetto, dei presupposti giustificanti
l’applicazione dell’art. 48, comma 2, Decreto Legislativo 31 marzo
2023 n. 36  in considerazione del valore presunto dell’affidamento; 

\textbf{CONSIDERATO~} che da preliminare e informale indagine
di mercato l’importo stimato del progettato affidamento si attesterà
certamente al di sotto della soglia di cui all’articolo 50 comma 1
lett. b) del Decreto Legislativo 31 marzo 2023, numero 36;

\textbf{RITENUTO~} di potere procedere, anche nell’ottica di
ottimizzare i tempi di realizzazione del suddetto progetto, mediante
affidamento diretto;

\textbf{CONSIDERATO~}  inoltre che da indagine prodromica
all’affidamento diretto, il valore presunto dello stesso sarà
sicuramente inferiore a 5.000,00 Euro;

\textbf{VISTO~} il Comunicato del Presidente ANAC
del 10 Gennaio 2024 con il quale "\textit{l’Autorità al fine di favorire
le Amministrazioni nell’adeguarsi ai nuovi sistemi che prevedono
l’utilizzo delle piattaforme elettroniche e garantire così un migliore
passaggio verso l’amministrazione digitale, sentito il Ministero
delle Infrastrutture e dei Trasporti, ritiene in ogni caso necessario
chiarire che allo scopo di consentire lo svolgimento delle ordinarie
attività di approvvigionamento in coerenza con gli obiettivi della
digitalizzazione, l’utilizzo dell’interfaccia web messa a disposizione
dalla piattaforma contratti pubblici - PCP dell’Autorità, raggiungibile
al link https://www.anticorruzione.it/-/piattaforma-contrattipubblici,
sarà disponibile anche per gli affidamenti diretti di importo inferiore
a 5.000 euro fino al 30 settembre 2024}"

\textbf{VISTO~} il Comunicato del Presidente ANAC del 28 giugno 2024
con il quale "è prorogata fino al 31 dicembre 2024 la possibilità di utiulizzare
l'interfaccia web messa a disposizione dalla Piattaforma PCP dell'autorità";

\textbf{RITENUTA~} l’opportunità di avviare detta
procedura di affidamento in conformità a quanto disposto nel Comunicato
del Presidente ANAC del 10 Gennaio 2024;

\textbf{RILEVATO~} che in data ((( pratica.data_doc )))
l’O.E., confermando l’offerta economica
già presentata in sede di indagine informale di mercato, inoltrava
la richiesta documentazione amministrativa, acquisita con numero di protocollo
((( pratica.num_protocollo_doc ))) in data ((( pratica.data_protocollo_doc )));

\textbf{VISTA~}l’offerta dell’Operatore
Economico, per complessivi Euro ((( pratica.costo_netto ))), al netto di IVA;

\textbf{CONSIDERATO~} che il Responsabile Unico del
Progetto, consultatosi preventivamente con il 
 Richiedente, ((( pratica.nome_richiedente ))), ha giudicato l’offerta congrua
 e valida;

\textbf{ACQUISITE~}   le dichiarazioni sostitutive
di atto di notorietà richieste dalla vigente normativa in materia di
appalti pubblici relativamente ad affidamenti di importo inferiore a
euro 40.000,00;

\textbf{VISTE~}	le risultanze a seguito delle verifiche avviate a
carico dell’Operatore Economico mediante acquisizione di "moduli" e
"dichiarazioni", debitamente compilati e sottoscritti dallo stesso, come
di seguito elencati e specificati: 

\begin{itemize}
	\item[$-$] "Richiesta di offerta", sottoscritta per presa visione ed
		    accettazione;

	\item[$-$] "Dichiarazioni amministrative";

	\item[$-$] "DGUE - Documento di Gara Unico Europeo"; 
\end{itemize}

\textbf{CONSIDERATO~} che, ai sensi dell’articolo 1 del
Decreto Legge 24 aprile 2017, numero 50, convertito, con modificazioni,
dalla Legge 21 giugno 2017, numero 96, che ha modificato l’articolo
17-ter del Decreto del Presidente della Repubblica 26 ottobre 1972, numero
633 (fatta eccezione per le prestazioni di servizi rese ai soggetti di
cui ai commi 1, 1-bis e 1-quinquies, i cui compensi sono assoggettati a
ritenute alla fonte a titolo di imposta sul reddito ovvero a ritenuta a
titolo di acconto di cui all' articolo 25 del decreto del Presidente della
Repubblica 29 settembre 1973, n. 600):

\begin{enumerate}

\item[a)] i pagamenti delle fatture
emesse a decorrere dal 1° luglio 2017 vengono effettuati al netto della
Imposta sul Valore Aggiunto e il relativo versamento deve essere eseguito
direttamente in favore dell’Erario;

\item[b)] la predetta procedura si
applica a tutte le Pubbliche Amministrazioni, ivi compresi gli "Enti
Pubblici di Ricerca", e, quindi anche allo "Osservatorio Astrofisico
Arcetri dello "Istituto Nazionale di Astrofisica";
\end{enumerate}

\textbf{VISTA~}	la Delibera del 20 dicembre 2023, numero 82, con la
quale il Consiglio di Amministrazione ha designato, tra gli altri,
il Dott. Simone Esposito, con decorrenza dal 01 gennaio 2024 e per la
durata di un triennio, quale Direttore dello "Osservatorio Astrofisico
di Arcetri";

\textbf{VISTO~}	il Decreto del Presidente del 21 dicembre 2023,
numero 32, con il quale, in attuazione della Delibera del Consiglio di
Amministrazione del 20 dicembre 2023, numero 82, e per i periodi temporali
in essa specificati, sono stati nominati, ai sensi dell’articolo
18 dello Statuto dello "Istituto Nazionale di Astrofisica", i nuovi
Direttori delle "Strutture di Ricerca;

\textbf{VISTO~}	inoltre, la Determina Direttoriale del 28 dicembre 2023,
numero 160, con la quale il Dottore Gaetano Telesio, nella sua qualità
di Direttore Generale dello "Istituto Nazionale di Astrofisica", ha
conferito, ai sensi dell’articolo 14, comma 3, lettera g), del vigente
Statuto, a decorrere dal 01 gennaio 2024 e per la durata di un triennio,
l'incarico di Direttore dello "Osservatorio Astrofisico di Arcetri"
al Dott. Simone Esposito;
   
\textbf{VISTA~}	la delibera di approvazione del Bilancio di previsione
dell’Istituto Nazionale di Astrofisica per l’esercizio finanziario
 in corso;

\textbf{ACCERTATA~}	la disponibilità finanziaria nei pertinenti
Capitoli di Spesa del predetto Bilancio;

\textbf{VISTA~}	la Legge del 13 agosto 2010 numero 136	e il
D.L. n. 187/2010 convertito nella Legge n. 217 del 17.12.2010, che
introducono l’obbligo di tracciabilità dei flussi finanziari relativi
alle commesse pubbliche;

\textbf{VISTA~}	la comunicazione dell’Agenzia delle Entrate del 28
giugno 2023,  Prot. n. 240013/2023,  avente ad oggetto	"\textbf{Individuazione
delle modalità telematiche di versamento dell’imposta di bollo, di cui
all’articolo 18, comma 10, del decreto legislativo 31 marzo 2023, n. 36,
che l’appaltatore assolve al momento della stipula del contratto}"

\textbf{VISTA~}	la Circolare dell’Agenzia delle Entrate del 22 luglio
2023 numero 22 avente ad oggetto "Articolo 18, comma 10, del decreto
legislativo 31 marzo 2023, n. 36, recante il Codice dei contratti pubblici
– Imposta di bollo".

\textbf{VISTA~}	la disponibilità sull’Ob. Fu. (Funzione Obiettivo)
in seguito richiamato in parte dispositiva;

\textbf{RITENUTO~} quindi che vi siano i presupposti normativi e
di fatto per procedere all’affidamento di quanto indicato in oggetto
ai sensi dell’Art. 50, comma 1, lett. b) del D.Lgs. 36/2023 e s.m.i.;

\textbf{VALUTATA~} la necessità di provvedere all’acquisizione
di quanto richiesto;

\begin{center}
	\textbf{DETERMINA}
\end{center}

Art. 1: Le premesse di cui al presente atto sono parte integrante e
sostanziale dello stesso e ne costituiscono le motivazioni ai sensi
dell’Art. 3 della Legge n. 241/90 e s.m.i.

Art. 2: che il/la Responsabile Unico/a di Progetto, in base all’Art. 15 del
D. Lgs 36/2023 e Allegato I.2 del suddetto decreto è ((( pratica.nome_rup ))),
che possiede le competenze necessarie a svolgere tale ruolo
e che, dal punto di vista tecnico scientifico, è affiancato/a da
((( pratica.nome_richiedente ))).

Art. 3: Di aggiornare e approvare il quadro economico come di seguito riportato:

((* include 'quadro_economico.tex' *))

A gravare su Ob.~Fu. numero ((( pratica.stringa_codf ))) – Responsabile dei
Fondi: ((( pratica.nome_responsabile ))).

Art. 4: Di autorizzare l’affidamento diretto della fornitura di ((( pratica.descrizione_acquisto ))) a
((( pratica.fornitore_nome ))) con sede in  ((( pratica.fornitore_sede )))
codice fiscale: ((( pratica.fornitore_codfisc ))) e partita IVA:
((( pratica.fornitore_partiva ))), per complessivi Euro ((( pratica.costo_netto))).

Art. 5: Di autorizzare la relativa spesa, per un importo complessivo
pari a ((( pratica.costo_totale ))) Euro comprensivo della Imposta sul
Valore Aggiunto, a carico della "Funzione Obiettivo" ((( pratica.stringa_codf))).
"Codice Unico Progetto" ("CUP"): ((( pratica.numero_cup )));  Capitolo:
((( pratica.capitolo ))).

Articolo 6: Di prenotare i correlativi impegni di spesa.

Articolo 7: Di autorizzare il pagamento a favore dell’affidatario,
mediante accredito sul conto corrente dedicato comunicato alla Stazione
Appaltante, del corrispettivo previsto per l’affidamento della
fornitura oggetto del presente provvedimento, che ammonta, al netto
della Imposta sul Valore Aggiunto, a Euro ((( pratica.costo_netto ))),
previa attestazione di regolare esecuzione.

Articolo 8: Di dare atto che avverso il presente procedimento è ammesso
ricorso giurisdizionale al Tribunale Amministrativo Regionale per la
Toscana nei termini previsto dall’articolo 120 del Decreto Legislativo
2 luglio 2010 n. 104 e successive modificazioni e integrazioni;

Articolo 9: Di dare atto che, rispetto alla procedura di affidamento in
oggetto, sia nei confronti del RUP che del Direttore dell’Osservatorio:

\begin{itemize}

\item[$-$]  non ricorre conflitto di interessi, neppure potenziale, ai
sensi dell’art. 6 bis della Legge n. 241/1990, dell’articolo 6 del
D.P.R. n. 62/2013 e del vigente codice di comportamento dell’Istituto
Nazionale di Astrofisica;

\item[$-$]  non ricorrono le condizioni di astensione di cui
all’articolo 14 del D.P.R. n. 62/2013 e di cui al vigente codice di
comportamento dell’Istituto Nazionale di Astrofisica;

\item[$-$] non ricorrono le situazioni di conflitto di interesse di cui
all’articolo 16 del Decreto Legislativo n. 36/2023.

\end{itemize}

((* include 'direttore.tex' *))
\end{document}
