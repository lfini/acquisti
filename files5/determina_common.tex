
%%  Parte comune a tutte le determine. Vers. 1.0

\vspace{0.5cm}
\begin{center}
Il Direttore
\end{center}

\begin{list}{VALUTATA}{}

\item[VISTA]
la Legge 7 agosto 1990, numero 241, e successive modifiche ed integrazioni, che contiene
\textbf{"Nuove norme in materia di procedimento amministrativo e di diritto di accesso ai
documenti amministrativi"}, ed, in particolare gli articoli 4, 5 e 6;

\item[VISTO]
il Decreto Legislativo 30 marzo 2001, numero 165, e successive modificazioni ed integrazioni, che contiene
\textbf{"Norme generali  sull'ordinamento del lavoro alle dipendenze delle amministrazioni pubbliche"}
ed, in particolare, gli articoli 1, 2, 4, 16 e 17;

\item[VISTO]
il Decreto del Presidente della Repubblica 27 febbraio 2003, numero 97, con il quale è stato emanato il
\textbf{"Regolamento concernente l'amministrazione e la contabilità degli enti pubblici di cui alla Legge
	20 marzo 1975, n. 70"}, ed, in particolare, gli articoli 30, 31 e 32; 

\item[VISTO]
il Decreto Legislativo 4 giugno 2003, numero 138, pubblicato nella Gazzetta Ufficiale della Repubblica Italiana,
Serie Generale, del 19 giugno 2003, numero 140, che disciplina il \textbf{"Riordino dello Istituto Nazionale
di Astrofisica"}, come modificato e integrato dallo \textbf{"Allegato 2"} del Decreto Legislativo 21 gennaio 2004,
numero 38, che, tra l'altro, istituisce, ai sensi dell'articolo 1 della Legge 6 luglio 2002, n. 137, lo
\textbf{"Istituto Nazionale di Ricerca Metrologica"};

\item[VISTO]
il Decreto Legislativo 30 giugno 2003, numero 196, con il quale è stato adottato il
\textbf{"Codice in materia di protezione dei dati personali"};

\item[VISTO]
il "Regolamento (UE) 2016/679 del Parlamento e del Consiglio Europeo del 27 aprile 2016, relativo
alla protezione delle persone fisiche con riguardo al trattamento dei dati personali, nonché alla
libera circolazione di tali dati, che abroga la Direttiva 95/46/CE", denominato anche
"Regolamento Generale sulla Protezione dei Dati" ("RGPD"), in vigore dal 24 maggio 2016 e
applicabile a decorrere dal 25 maggio 2018;

\item[VISTO]
il Decreto Legislativo 10 agosto 2018, numero 101, che contiene alcune Disposizioni per l’adeguamento
della normativa nazionale alle disposizioni del Regolamento (UE) 2016/679 del Parlamento e del Consiglio
Europeo del 27 aprile 2016, relativo alla protezione delle persone fisiche con riguardo al trattamento
dei dati personali, nonché alla libera circolazione di tali dati, che abroga la Direttiva 95/46/CE",
denominato anche "Regolamento Generale sulla Protezione dei Dati" ("RGPD");

\item[VISTA]
la Legge del 13 agosto 2010 n.136 e successive modifiche ed integrazioni ed in particolare l’Art. 3
che introduce l'obbligo di tracciabilità dei flussi finanziari relativi alle commesse pubbliche;

\item[VISTO]
il Decreto Legge 6 luglio 2011, numero 98, che contiene \textbf{"Disposizioni urgenti per la stabilizzazione
finanziaria"}, convertito, con modificazioni, dalla Legge 15 luglio 2011, numero 111, ed, in particolare,
l'articolo 11, che disciplina gli \textbf{"Interventi per la razionalizzazione dei processi di
approvigionamento di beni e servizi della \underline{Pubblica Amministrazione}"}, e \underline{che dispone},
tra l'altro, che, qualora \textit{"...non si ricorra alle convenzioni di cui all'articolo 1, comma 449,
della Legge 27 dicembre 2006, numero 296, gli atti e i contratti posti in essere in violazioni delle
disposizioni sui parametri contenuti nell'articolo 26, comma 3, della Legge 23 dicembre 1999, numero 488,
sono nulli e costituiscono illecito disciplinare e determinano responsabilità erariale ...}

\item[VISTO]
il Decreto Legge 7 maggio 2012,  numero 52,  che contiene \textbf{"Disposizioni urgenti per la
razionalizzazione della spesa pubblica"}, convertito, con modificazioni, dalla Legge 6 luglio 2012,
numero 94, ed, in particolare, l'articolo 7, che ha modificato l'articolo 1, commi 449 e 450, della
Legge del 27 dicembre 2006, numero 296, prevedendo, tra l'altro, che: nel rispetto nel rispetto del
\textit{"... sistema delle convenzioni di cui agli articoli 26 della Legge 23 dicembre 1999, numero 488
e successive modificazioni, e 58 della Legge 23 dicembre 2000, numero 388, tutte le amministrazioni
statali centrali e periferiche, ivi compresi gli istituti e le scuole di ogni ordine e grado, le istituzioni
educative e le istituzioni universitarie, nonché gli enti nazionali di previdenza e assistenza sociale
pubblici e le agenzie fiscali di cui al Decreto Legislativo  30 luglio 1999,  numero 300, sono  tenute}
\begin{itemize}
\item \textit{ad approvigionarsi utilizzando le Convenzioni Quadro..." ; le "... amministrazioni statali centrali
e periferiche, ad esclusione degli istituti e delle scuole di ogni ordine e grado, delle istituzioni
educative e delle istituzioni universitarie, nonché gli enti nazionali di previdenza e di assistenza sociale
pubblici e le agenzie fiscali di cui al Decreto Legislativo 30 luglio 1999, n. 300, per gli acquisti di beni
e servizi di importo pari o superiore a 1.000 euro e al di sotto della soglia di rilievo comunitario, sono
tenute a fare ricorso al "Mercato Elettronico della Pubblica Amministrazione" di cui all'artico/o 328,
comma 1, del Regolamento emanato con Decreto del Presidente della Repubblica 5 ottobre 2010, numero 207...";}

\item \textit{fermi restando "...gli obblighi e le facoltà previsti al comma 449 del presente articolo, le altre
amministrazioni pubbliche di cui all'articolo 1 del Decreto Legislativo 30 marzo 2001, numero 165, nonché
le autorità indipendenti, per gli acquisti di beni  e servizi di importo pari o superiore a 1.000 euro e
inferiore alla soglia di rilievo comunitario sono tenute a fare ricorso al "Mercato Elettronico della
Pubblica  Amministrazione" owero ad altri mercati elettronici istituiti ai sensi del medesimo articolo 328
ovvero  al sistema  telematico  messo a disposizione dalla centrale regionale di riferimento per lo
svolgimento delle relative procedure ..." ;}
\end{itemize}

\item[VISTO]
il Decreto Legislativo 18 aprile 2016, numero 50,e successive modifiche ed integrazioni  con il quale: 
\begin{itemize}
\item è stata data piena attuazione alle Direttive della Unione Europea numeri 2014/23/UE, 2014/24/UE
e 2014/25/UE, le quali:
\begin{itemize}
\item hanno "modificato" la disciplina vigente in materia di "aggiudicazione dei contratti di concessione,
di appalti pubblici e di procedure di appalto degli enti erogatori nei settori dell'acqua, dell'energia,
dei trasporti e dei servizi postali 
\item hanno  "riordinato" la "disciplina vigente in materia di contratti pubblici relativi a lavori, servizi
e forniture";
\end{itemize}
\item è stato adottato, a tal fine, il nuovo  "Codice  degli  Appalti  Pubblici e dei Contratti di Concessione";
\end{itemize}

\item[VISTI]
in, particolare, gli articoli 35 e 36 del Decreto Legislativo 18 aprile 2016, numero 50, e successive modifiche
ed integrazioni;

\item[VISTO]
il Decreto Legge 16 luglio 2020, n. 76 \textit{“Misure urgenti per la semplificazione e l'innovazione digitale”}
convertito con modificazioni dalla L 11 settembre 2020, n. 120 ed in particolare gli articoli dall’1 al 8;

\item[VISTO]
Decreto Legge 31 maggio 2021, n. 77, convertito, con modificazioni, dalla L. 29 luglio 2021, n. 108;
ed in particolare l'art. 51, comma 1, lett. a), n. 2.2), ed il comma 3;

\item[VISTO]
il Decreto del Presidente della Repubblica 5 ottobre 2010, numero 207, e successive modifiche ed integrazioni,
con il quale è stato emanato il \textbf{"Regolamento di esecuzione  e  di  attuazione  del Decreto Legislativo
12 aprile 2006, n. 163, recante il Codice dei contratti pubblici relativi a lavori, servizi e forniture in
attuazione delle direttive 2004117/CE e 2004/18/CE''}, limitatamente alle disposizioni normative non abrogate
a seguito della entrata in vigore del Decreto Legislativo 18 aprile 2016, numero 50;

\item[PRESO ATTO]
delle linee guida dell'ANAC;

\item[VISTO]
il Decreto Legislativo 25 novembre 2016 n. 218 “Semplificazione delle attivita' degli enti  pubblici
di  ricerca sensi dell'articolo 13 della legge 7 agosto 2015, n. 124” ed in particolare l'Art. 10;

\item[VISTO]
il \textbf{"Piano per la Informatica nella Pubblica Amministrazione per il Triennio 2017 2019"},
approvato con Decreto del Presidente del Consiglio dei Ministri del 31 maggio 2017;

\item[VISTO]
lo Statuto dell’INAF, adottato dal CdA con delibera n. 42 del 25 maggio 2018 ed entrato in vigore il
24 settembre 2018 e successive modifiche ed integrazioni;

\item[VISTO]
il Regolamento sull’Amministrazione, sulla Contabilità e sull’Attività Contrattuale dell’INAF, pubblicato
sul S.O. n. 185 alla G.U. Serie Generale n. 300 del 23 dicembre 2004, in particolare nel suo articolo 14
come modificato al comma 4 con Delibera n. 100 del 8 novembre 2005 pubblicata sulla G.U. n. 31 serie generale
del 7/2/2006 e con Delibera n. 46 del 2 luglio 2009, approvata dal Ministero dell’Istruzione, dell’Università
e della Ricerca con nota prot. n. 628 del 29 luglio 2009;

\item[VISTO]
il Regolamento di Organizzazione e Funzionamento dell’INAF (ROF), approvato dal Consiglio di Amministrazione
con delibera del 5 giugno 2020 n. 46 e successive modifiche ed integrazioni;  

\item[VISTO]
il Decreto del Presidente n. 14 del 30 dicembre 2020 con il quale sono stati nominati alcuni Direttori delle
strutture territoriali INAF a decorrere dal 1 gennaio 2021 per la durata di un triennio;

\item[CONSIDERATO]
che con Determinazione del Direttore Generale n.188/2020 del 30 dicembre 2020 sono stati conferiti gli
incarichi dei Direttori delle strutture territoriali INAF e che alla scrivente è stato conferito l’incarico
di direttore dell’Osservatorio Astrofisico di Arcetri;

\item[VISTA]
la delibera del Consiglio di Amministrazione di approvazione del bilancio di previsione dell’INAF per
l’esercizio finanziario in corso;

