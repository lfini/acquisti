\documentclass[a4paper,12pt]{letter}
\usepackage[T1]{fontenc} 
\usepackage[utf8]{inputenc}
\usepackage[italian]{babel} 
\usepackage{eurosym}

%% template: ordine_italiano.tex. Vers. 1.6

((* include 'header.tex' *))

\begin{document}
\topaddr

{\small Pratica N. ((( pratica.numero_pratica ))) }
\vspace{.4cm}

\begin{flushright}
\begin{minipage}{12cm}
Spett.le:\\
\hspace*{0.5cm}{\bf ((( pratica.nome_fornitore ))) \\
\hspace*{0.5cm}((( pratica.ind_fornitore ))) }
\end{minipage}
\end{flushright}
\vspace{5mm}

CIG: ((( pratica.cig ))) 

CUP: ((( pratica.cup )))

Codice Univoco Ufficio (CUU): 14CVDG

BUONO D'ORDINE N. ~~ ((( pratica.numero_ordine ))) ~~ DEL ~~ ((( pratica.data_ordine )))

((* if pratica.dir_is_m *))
Il sottoscritto
((* else *))
La sottoscritta
((* endif *))
((( pratica.nome_direttore ))), 
nella veste di Direttore dell'INAF - Osservatorio Astrofisico di Arcetri,
con il presente atto affida ai sensi dell'art. 50 del D.Lgs. 36/2023 e s.m.i.
la fornitura~/~il servizio~/~il lavoro di seguito specificata/o:

\begin{quote}
((( pratica.descrizione_ordine )))
\end{quote}

da effettuarsi al prezzo di ((( pratica.str_costo_ord_it ))).

((* if pratica.note_ordine *))
{\bf Note:} ((( pratica.note_ordine )))
((* endif *))
\vspace{1cm}

{\tiny\bf NB.: \\
1) Questo Ente è sottoposto alla disciplina dello split payement. \\
2) E' necessario riportare in fattura sia il CIG che il CUP, se presenti,
nonché il ns. Codice Fiscale 97220210583, e la Partita IVA 06895721006.
La fattura dovrà essere emessa solo successivamente alla consegna del bene o all’esecuzione del servizio.\\
3) L'imposta di bollo è a carico del fornitore.

Si informa che, in ottemperanza alle disposizioni contenute nell'art.~3, Legge 13 agosto 2010, n.~136,
``Piano straordinario contro le mafie, nonch\'e delega al Governo in materia di normativa antimafia''
e s.m.i., qualora codesta Spett.le Ditta non utilizzasse il conto corrente dedicato 
(anche in via non esclusiva) alle commesse pubbliche per i movimenti finanziari 
relativi al presente contratto, lo stesso sarà risolto di diritto secondo quanto 
disposto dall'art.~3, comma~8 della legge n.~136/2010 e s.m.i..

Pagamento Bonifico Bancario: 30 gg. data acquisizione DURC.

Ai fini di un sollecito pagamento, si prega di
indicare in fattura il numero dell'ordinazione e non cumulare in un'unica fattura 
ordinazioni diverse;
indicare nella fattura il codice IBAN per il pagamento su C/C postale o bancario 
che dovrà coincidere con uno di quelli già comunicati quali conti dedicati alle commesse pubbliche.

Per chiarimenti interpellare: ((( pratica.nome_richiedente ))).
}

((* if pratica.dettaglio_ordine *))
\vspace{0.5cm}
\begin{flushright}
Segue allegato ...
\end{flushright}
%%% --ALLEGATI-- 

\newpage
\vspace{2cm}
\quad
((* endif *))

\vspace{1cm}

\begin{minipage}{\textwidth}
\begin{minipage}[t]{8cm}
\begin{center}
L'assegnatario dei fondi \\
~~ (((( pratica.nome_responsabile )))) ~~ \\
\end{center}
\end{minipage}\hfill\begin{minipage}[t]{8cm}
\begin{center}
Il Direttore \\
({}((( pratica.titolo_direttore ))) ((( pratica.nome_direttore )))) \\
\end{center}
\vspace*{25mm}
\end{minipage}
\begin{minipage}{\textwidth}
\begin{center}
{\bf Cod. Fisc. 97220210583 \hspace{1.5cm} Partita IVA 06895721006} \\
Consegna merce e fatture in Largo E. Fermi, 5 50125 Firenze
\end{center}
\end{minipage}
\end{minipage}
\end{document}
