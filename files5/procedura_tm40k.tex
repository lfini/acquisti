# Testo procedura per trattativa mepa sotto 40 k€

Affidamento diretto ai sensi del combinato disposto dagli articoli 17 comma 2 e 50, comma 1 lett. b) D.Lgs. 36/2023.\\
L’esplorazione informale di mercato sin qui condotta ha consentito infatti di individuare:
\begin{itemize}
\item un importo stimato del progettato affidamento sicuramente al di sotto della soglia contemplata dalla norma innanzi richiamata;
\item l’assenza di una Convenzione in ambito Consip;
\item presenza su MePA di almeno una ditta di riferimento che assicurasse le condizioni (tecniche e di budget) ritenute ottimali;
\item una ditta in grado di rispondere efficacemente al quadro esigenziale connesso al presente affidamento,
individuata sul mercato all’esito di una informale esplorazione e specificamente:
((( pratica.fornitore_nome ))) con sede legale in ((( pratica.fornitore_sede ))),
Cod.Fisc.: (((pratica.fornitore_codfisc ))), Part.IVA: ((( pratica.fornitore_partiva ))),
con la quale avviare una procedura di affidamento su piattaforma ''MePA''.
\end{itemize}

Si richiama inoltre l’art. 10, comma 3, D. Lgs. 218/2016 a mente
del quale ''Le disposizioni di cui all'articolo 1, commi 450, primo
periodo, e 452, primo periodo, della legge 27 dicembre 2006, n. 296, non
si applicano agli Enti per l'acquisto di beni e servizi funzionalmente
destinati all'attività di ricerca''.


